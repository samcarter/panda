%%%%%%%%%%%%%%%%%%%%%%%%%%%%%%%%%%%%%%%%%%%%%%%%%%%%%%%%%%%%%%%%%%%%%%
%
% Documentation for the panda package
% A package to draw bricks with tikz
% Maintained by samcarter
%
% Project repository and bug tracker:
% https://github.com/samcarter/panda
%
% Released under the LaTeX Project Public License v1.3c or later
% See https://www.latex-project.org/lppl.txt
%
%%%%%%%%%%%%%%%%%%%%%%%%%%%%%%%%%%%%%%%%%%%%%%%%%%%%%%%%%%%%%%%%%%%%%%
% !TeX program = txs:///arara
% arara: latexmk: {
% arara: --> engine: pdflatex,
% arara: --> options: [
% arara: -->    '-shell-escape',
% arara: -->    '-synctex=1',
% arara: -->    '-interaction=nonstopmode',
% arara: -->  ]
% arara: --> }
\documentclass{scrartcl}

\usepackage[
  themecolor=samdgray
]{\jobname-settings}

% meta %%%%%%%%%%%%%%%%%%%%%%%%%%%%%%%%%%%%%%%%%%%%%%%%%%%%%%%%%%%%%%%
\title{The panda package}
\subtitle{Estimating the blackness of fonts}
\author{%
  \texorpdfstring{
   \begin{tikzpicture}
     \panda
   \end{tikzpicture}\\[0.8em]
   \texttt{samcarter}\\
   \url{https://github.com/samcarter/panda}\\
%   \url{https://ctan.org/pkg/panda}
  }{samcarter}}
\date{Version v0.0 \textendash{} 2025/09/26}
\packagename{panda}

\colorlet{red}{samred}
\colorlet{blue}{samlblue}
\colorlet{orange}{samorange}
\colorlet{teal}{samteal}

\usepackage{busyPanda}
\usepackage{tikzlings}

\newcommand{\bamboo}[2]{%
  \rule{#1}{#2}\,\rule{#1}{#2}\,\rule{#1}{#2}%
}

\begin{document}
\maketitle
\thispagestyle{scrheadings}

\section{Introduction}
\label{intro}

Imagine the following problem:
there is a hungry panda sitting next to you and just today, you happen to not have any bamboo with you.
Now you are a \LaTeX\ user and the only limits are your own imagination, so of course you try to draw a couple of bamboo sticks with \LaTeX.
Here dear panda, have three bamboo sticks: \bamboo{0.8pt}{\fontcharht\font`I}\,.
The panda likes them and wants to make more.
{\fontfamily{ugq}\selectfont However the panda uses a different font and the bamboo sticks suddenly look a lot less tasty, they look just too thin: \bamboo{0.8pt}{\fontcharht\font`I}\,.}

The panda package provides estimates for the stroke width of fonts and how black the fonts look on the page.
Creators of symbols etc.\ can use these estimates to adjust their drawings to better blend in with the surrounding text.

{\fontfamily{ugq}\selectfont Tasty bamboo sticks: \bamboo{\busyPanda{0.1}\fontcharht\font`I}{\fontcharht\font`I}}

Note that this package was developed and tested with texts in Latin scripts -- results might be less satisfactory if used with other scripts.

\blurb*

\section{Clever Panda}

The clever panda approach provides an estimage

\section{Busy Panda}

The approach of the busy panda is different.
To get an estimate of how black text in a particular font looks like, I'm producing a sample pdf with:
\begin{center}
\ttfamily Lorem ipsum dolor sit amet, consectetur adip
\end{center}
After converting the pdf into a pixel graphic, I can calculate the average blackness of the pixel.
After repeating this for different fonts (including their bold versions), I now have a look-up table with the average blackness for 8800 fonts, normalised to the average blackness of computer modern roman.
Tests with different sample texts and image resolutions indicate that the error of this method is about 10\%.

\subsection{Usage}

\begin{tcblisting}{
  title={Usage},
  compilable listing sam,
  comment={
    \includegraphics{\jobname-listing-\thetcblistingcount}
  },
}
\documentclass[varwidth,border=1pt]{standalone}

\usepackage{busyPanda}

\newcommand{\bamboo}{%
  \rule{%
    \busyPanda{0.1}\fontcharht\font`I
  }{%
    \fontcharht\font`I
  }
}

\begin{document}

normal cmr: \bamboo \par
\bfseries bold cmr: \bamboo \par
\normalfont \fontfamily{cmbr}
normal cmbr: \bamboo \par
\bfseries bold cmbr: \bamboo
\end{document}
\end{tcblisting}

\subsection{Unsupported fonts}

...

\end{document}
